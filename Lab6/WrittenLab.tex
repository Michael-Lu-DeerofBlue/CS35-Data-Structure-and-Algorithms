% This LaTeX file contains your written lab questions.  You may answer
% these questions just by inserting your answer into this document.
%
% If you're unfamiliar with LaTeX, see the document LearningLaTeX.tex
% in this same directory.  It contains a brief explanation and a few
% snippets of LaTeX code to get you started; in fact, it should have
% everything you need to complete this assignment.
\documentclass{article}

\usepackage{amsmath}
\usepackage{amssymb}
\usepackage{amsthm}
\usepackage{algpseudocode}
\usepackage{algorithmicx}
\usepackage{alltt}
\usepackage{enumerate}

\newtheorem{claim}{Claim}

\begin{document}
    \section{Inductive Proofs}

    Prove each of the following claims by induction

    \begin{claim}
      The sum of the first $n$ even numbers is $n^2 + n$.  That is, $\sum\limits_{i=1}^{n} 2i = n^2 + n$.
    \end{claim}
    \begin{proof}

      Base case $i = 1$
      \begin{align*} 
        LHS &=  2 * 1 \\ 
         &=  2\\
        RHS &= 1^2+1\\
        &=2\\
        LHS &= RHS\\
        \end{align*}
    The base case is established.

    Inductive hypothesis: Assume that
      
      \[ \sum\limits_{i=1}^{n} 2i = n^2 + n \mbox{ for all $1 \leq n \leq k$}\]
     
    Inductive step: Show that
      \begin{align*} 
        \sum\limits_{i=1}^{k+1} 2i &= \sum\limits_{i=1}^{k} 2i + 2(k+1)\\ 
         &=  k^2+k+2k+2 \mbox{ This is true on our Inductive hypothesis}\\
         &= k^2+2k+1+k+1\\
         &= (k+1)^2 + (k+1)
        \end{align*}
    We have shown that $\sum\limits_{i=1}^{k+1} 2i = (k+1)^2 + (k+1)$.
  
  \end{proof}
    \begin{claim}
      $\sum\limits_{i=1}^{n} \frac{1}{2^i} = 1 - \frac{1}{2^n}$
    \end{claim}

    \begin{proof}

      Base case $i = 1$
      \begin{align*} 
        LHS &=  \frac{1}{2^1} \\ 
         &=  \frac{1}{2}\\
        RHS &= 1- \frac{1}{2^1}\\
        &=\frac{1}{2}\\
        LHS &= RHS\\
        \end{align*}
    The base case is established.

    Inductive hypothesis: Assume that
      
      \[ \sum\limits_{i=1}^{n} \frac{1}{2^i} = 1 - \frac{1}{2^n} \mbox{ for all $1 \leq n \leq k$}\]
     
    Inductive step: Show that
      \begin{align*} 
        \sum\limits_{i=1}^{k+1} \frac{1}{2^i} &= \sum\limits_{i=1}^{k} + \frac{1}{2^{k+1}}\\ 
         &=  1 - \frac{1}{2^k} + \frac{1}{2^{k+1}} \mbox{ This is true on our Inductive hypothesis}\\
         &= 1 + \frac{-2}{2^{k+1}} + \frac{1}{2^{k+1}}\\
         &= 1-\frac{1}{2^{k+1}}
        \end{align*}
    We have shown that $\sum\limits_{i=1}^{n} \frac{1}{2^i} = 1 - \frac{1}{2^n}$.
  
  \end{proof}


    \begin{claim}
      $\sum\limits_{i=0}^{n} 2^i = 2^{n+1} - 1$
    \end{claim}

    \begin{proof}

      Base case $i = 0$
      \begin{align*} 
        LHS &=  2^0 \\ 
         &=  1\\
        RHS &= 2^{0+1}-1\\
        &= 2-1\\
        &= 1\\
        LHS &= RHS\\
        \end{align*}
    The base case is established.

    Inductive hypothesis: Assume that
      
      \[ \sum\limits_{i=0}^{n} 2^i = 2^{n+1} - 1 \mbox{ for all $0 \leq n \leq k$}\]
     
    Inductive step: Show that
      \begin{align*} 
        \sum\limits_{i=0}^{k+1} 2^i &= \sum\limits_{i=0}^{k} 2^i + 2^{k+1}\\ 
         &=  2^{k+1} - 1 +  2^{k+1}\mbox{ This is true on our Inductive hypothesis}\\
         &= 2*2^{k+1} - 1\\
         &= 2^{(k+1)+1} - 1
        \end{align*}
    We have shown that $\sum\limits_{i=0}^{n} 2^i = 2^{n+1} - 1$.
  
  \end{proof}


    \vspace{1cm}
    \section{Recursive Invariants}
    
    The function \texttt{minEven}, given below in pseudocode, takes as
    input an array $A$ of size $n$ of numbers.  It returns the
    smallest \emph{even} number in the array.  If no even numbers
    appear in the array, it returns positive infinity ($+\infty$).
    Using induction, prove that the \texttt{minEven} function works
    correctly.  Clearly state your recursive invariant at the
    beginning of your proof.

    \begin{alltt}
Function minEven(A,n)
  If n is 0 Then
    Return +infinity
  Else
    Set best To minEven(A,n-1)
    If A[n-1] < best And A[n-1] is even Then
      Set best To A[n-1]
    EndIf
    Return best
  EndIf
EndFunction
    \end{alltt}

    \begin{proof}
      P(n) = the function minEven(A,n) returns the smallest even number in the frist n values of the array, otherwise it
      returns positive infinity.

      Base case $n = 0$
      
      The function returns positive infinity when $n=0$. This is true because there is no value inside the array.
      The base case is established.

    Inductive hypothesis: Assume that the function minEven is correct for all arrays of size n where $0 \leq n \leq k.$
     
    Inductive step: Show that the function minEven is correct for all arrays of size k+1.
    
    minEven(A,k+1)

    case 2

    Set the variable best to the result of minEven(A,k). We know minEven(A,k) returns the smallest even number in the frist k values of the array, otherwise it
    returns positive infinity base on the inductive hypothesis.

    case 2a

    If A[k] is less than the best and A[k] is even, then it set best to A[k]. The function returns best.
    
    case 2b

    If A[k] is greater than best or A[k] is odd, then best remains the same. The function returns best. 

    The function has compared the element A[k+1] to minEven(A,k). We know that minEven(A,k) returns the smallest even number in the frist n values of the array, otherwise it
    returns positive infinity through the inductive hypothesis. Thus, minEven(A,k+1)
    will return the smallest even number in the frist k+1 values of the array, otherwise it
    returns positive infinity
  
  \end{proof}

\end{document}
