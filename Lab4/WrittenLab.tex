% This LaTeX file contains your written lab questions.  You may answer
% these questions just by inserting your answer into this document.
%
% If you're unfamiliar with LaTeX, see the document LearningLaTeX.tex
% in this same directory.  It contains a brief explanation and a few
% snippets of LaTeX code to get you started; in fact, it should have
% everything you need to complete this assignment.

\documentclass{article}

\usepackage{amsmath}
\usepackage{amssymb}
\usepackage{amsthm}
\usepackage{graphicx}
\usepackage{listings}
\usepackage{color}
\usepackage{enumitem}

\definecolor{dkgreen}{rgb}{0,0.6,0}
\definecolor{gray}{rgb}{0.5,0.5,0.5}
\definecolor{mauve}{rgb}{0.58,0,0.82}

\lstset{frame=tb,
  aboveskip=3mm,
  belowskip=3mm,
  showstringspaces=false,
  columns=flexible,
  basicstyle={\small\ttfamily},
  numbers=none,
  numberstyle=\tiny\color{gray},
  keywordstyle=\color{blue},
  commentstyle=\color{dkgreen},
  stringstyle=\color{mauve},
  breaklines=true,
  breakatwhitespace=true,
  tabsize=3
}


\begin{document}

    \section{QuickSort}

    See the source code file \texttt{quickSort.cpp} and the tests
    given in \texttt{tests.cpp}. No written response is needed for
    this part of the lab.

    \section{Big-O Proofs}

    \vspace{2mm}
    \noindent {\large\textbf{Problem 1.}} Show that $8n^3+7n^2-12$ is $O(n^3)$.

\begin{proof}
    Assume $n>0$ and $n\in\mathbb{Z}$. Observe

    $$
    \begin{cases}
      8n^3\leq 8n^3\\
      7n^2\leq 7n^3\\
      -12\leq 0\\  
    \end{cases}
    $$

    Notice $8n^3+7n^2-12\leq 8n^3+7n^3=\underbrace{15}_{c}n^3$.
    Therefore, $8n^3+7n^2-12$ is $O(n^3)$ when $n_0=1, c=15$.

\end{proof}

    \vspace{1cm}
    \noindent {\large\textbf{Problem 2.}} Show that $6n^2-n+4$ is $O(n^2)$.

    \begin{proof}
        Assume $n>0$ and $n\in\mathbb{Z}$. Observe
    
        $$
        \begin{cases}
          6n^2\leq 6n^2\\
          -n\leq 0\\
          4\leq 4n^2\\  
        \end{cases}
        $$
    
        Notice $6n^2-n+4\leq 6n^2+4n^2=\underbrace{10}_{c}n^2$.
        Therefore, $6n^2-n+4$ is $O(n^2)$ when $n_0=1, c=10$.
    
    \end{proof}

    \newpage
    \vspace{1cm}
    \section{Mystery Functions}

    %%%%%%%%%%%%%%
    %%%% TODO %%%%
    %%%%%%%%%%%%%%
    % Give your Big-O analysis of each of the mystery functions here.
\begin{enumerate}[label=\Alph*]
    \item \begin{lstlisting}
    Function fnA(n):
        For i In 1 To n/2:
            Set a To i
        EndFor
    EndFunction
    \end{lstlisting}

        We claim that \lstinline|fnA(n)| is $O(n)$. Inside the first loop there is only one step of constant time, and the bigger loop has $\frac{n}{2}$ steps. 
        Thus, the program can be generalized as $O(n)$.

    \item \begin{lstlisting}
    Function fnB(n):
        For i In 1 to n:
            For j In 1 to n:
                Set a To i
            EndFor
        EndFor
    EndFunction
    \end{lstlisting}

    We claim that \lstinline|fnB(n)| is $O(n^2)$. Observe the program consists of a nested loop, while the outer loop executes $n$ times, the inner loop executes $n$ times. The command inside the inner loop is only one step of constant time. Thus, the outer loop executes $n$ times over the inner loop that runs for $n$ times, which conforms with $O(n^2)$.

    \item \begin{lstlisting}
    Function fnC(n):
        For i In 1 to n:
            Set j To 1
            While j < n:
                Set j To j*2
            EndWhile
        EndFor
    EndFunction
    \end{lstlisting}

    We claim that \lstinline|fnC(n)| is $O(n\log n)$. The program is complicated by a for loop that runs $n$ times and a while loop inside. We interpret that the while loop exits after $\log_2 n$ times, because each iteration the variable \lstinline|j| approaches \lstinline|n| doubly faster than the last iteration. As \lstinline|i| gets larger linearly by virtue of the for loop, it takes less time for the while loop to end.

    \newpage

    \item \begin{lstlisting}
    Function fnD(n):
        For i In 1 to n*n:
            For j In 1 to n*n:
                Set a To j
            EndFor
        EndFor
    EndFunction
    \end{lstlisting}

    We claim that \lstinline|fnD(n)| is $O(n^4)$. The program is quite akin to \lstinline|fnB(n)|, while the only difference is that both the inner and the outer loop takes $n^2$ steps instead. Consequently, the overall execution would take a $n^2 \cdot n^2=n^4$ time.


    \item \begin{lstlisting}
    Function fnE(n):
        For j In 1 To 4:
            Set a To i
        EndFor
    EndFunction
    \end{lstlisting}

    We claim that \lstinline|fnE(n)| is $O(1)$. Whatever $n$ is does not affect the running time of the program, given that $n$ is not involved in the execution of the for loop, which indicates that the program is of constant time. 

    \item \begin{lstlisting}
    Function fnF(n):
        Set i To 0
        While i < n*n*n:
            Set i To i + 1
        EndWhile
    EndFunction
    \end{lstlisting}

    We claim that \lstinline|fnF(n)| is $O(n^3)$. Since \lstinline|i| always starts with a value of $0$, the while loop should always take \lstinline|n*n*n| folds.
\end{enumerate}


    % After your explanations, list the mystery functions A-E in sorted
    % order from fastest to slowest.

    We claim the sorted order of the functions from fastest to slowests is

    $$
    E<A<C<B<F<D
    $$
    
\end{document}
